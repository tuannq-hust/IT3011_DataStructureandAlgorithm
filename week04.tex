\documentclass[aspectratio=169]{beamer}
\usetheme[theme=blue,logo=logowithtextvi]{HUST} 
\DeclareUnicodeCharacter{221E}{\ensuremath{\infty}}

\usepackage[T5]{fontenc}
\usepackage[utf8]{inputenc}
% \usepackage[utf8]{vietnam} % Uncomment nếu cần thiết tuỳ hệ thống
\usepackage{amsmath}
\usepackage{amsfonts}
\usepackage{amssymb}
\usepackage{graphicx}
\usepackage{adjustbox}
\usepackage{xcolor}
\usepackage{tikz}
\usetikzlibrary{positioning,calc}
\definecolor{codeblue}{RGB}{0,90,200}   % xanh nhạt
\definecolor{codegold}{RGB}{210,160,0}

\usepackage{minted}

\setminted{
	breaklines=false,
	autogobble=false,
	obeytabs=true,
	tabsize=2,
	linenos=true,
	showspaces=false,
	space=~,
	baselinestretch=1,
	fontsize=\normalsize
}

\newcommand{\placecontent}[4]{%
  \tikz[remember picture,overlay]
    \node[anchor=north west]
      at ([xshift=#1,yshift=-#2]current page.north west)
      {\parbox{#3}{#4}};
}

\graphicspath{{./week04_resources/}}

\title{\huge CẤU TRÚC DỮ LIỆU VÀ GIẢI THUẬT}
\author{SoICT - HUST}
\date{}

\begin{document}

% 2 slides đầu tiên:
\HUSTInsertBrandSlide
\HUSTInsertThemeSlide

% Slide tiêu đề
{\HUSTUseBackground{onelove.pdf}
\begin{frame}
  \ifdefstring{\insertaspectratio}{169}{
    \HUSTCornerImage[0.14]{assets/logo/soict_vi_h.pdf}
    \placecontent{0.5cm}{0.33\paperheight}{0.85\paperwidth}{
        \color{\HUSTFrameTitleTextColor}\bfseries\fontsize{22pt}{30pt}\selectfont
        \inserttitle
    }
    \placecontent{0.5cm}{0.50\paperheight}{0.8\paperwidth}{
        \color{\HUSTFrameTitleTextColor}\fontsize{14pt}{14pt}\selectfont
        %Bài học
        \textbf{\large TUẦN 4: NHÁNH VÀ CẬN}\\
    }
  }{}
\end{frame}
}

% Outline
\AtBeginSection[]
{
    \begin{frame}<beamer>
        \frametitle{NỘI DUNG}
        \tableofcontents[currentsection]
    \end{frame}
}

%Nội dung chính trong slides

\section{Sơ đồ chung nhánh và cận}
\begin{frame}[t]{SƠ ĐỒ CHUNG} % [t] = canh nội dung lên phía trên
	\setlength{\leftmargini}{-1.2em}
	
	\begin{itemize}
		\item Nhánh và cận (Branch and Bound): một trong số các phương pháp để
		giải bài toán tối ưu tổ hợp
		\begin{itemize}
			\item Dùng kỹ thuật quay lui để liệt kê tất cả các phương án,
			từ đó giữ lại phương án tốt nhất
			\item Dùng đánh giá cận (cận trên với bài toán tìm max và cận dưới
			với bài toán tìm min) để cắt bớt không gian tìm kiếm
			trong quá trình liệt kê
		\end{itemize}
	\end{itemize}
\end{frame}




\begin{frame}[t]{SƠ ĐỒ CHUNG}
	\begin{columns}[T,onlytextwidth]
		% Cột bên trái: chữ
		\begin{column}{0.47\textwidth}
			\setlength{\leftmargini}{-1.2em} % (tuỳ, để bullet đỡ thụt sâu)
			\begin{itemize}
				\item Xét bài toán tìm cực tiểu của hàm mục tiêu, trong đó
				lời giải được biểu diễn bởi một bộ các biến:
				\[
				X = (X_1, X_2, \dots, X_n).
				\]
				\item Hàm \texttt{Try(k)} dùng để thử giá trị cho biến $X_k$
				trong quá trình liệt kê.
				\item Ký hiệu $f^*$: giá trị hàm mục tiêu của phương án
				tốt nhất đã tìm được.
			\end{itemize}
		\end{column}
		
		% Cột bên phải: hình cây nhánh và cận
		\begin{column}{0.53\textwidth}
			\centering
			\includegraphics[width=1.15\linewidth]{branch_and_bound_tree.png}
			% hoặc tên file đúng với file bạn lưu
		\end{column}
	\end{columns}
\end{frame}


\begin{frame}[t]{SƠ ĐỒ CHUNG}
	\begin{columns}[T,onlytextwidth]
		% Cột bên trái: text
		\begin{column}{0.47\textwidth}
			\setlength{\leftmargini}{-1.2em}
			\begin{itemize}
				\item Sau khi gán giá trị $v$ cho $X_k$, ta đánh giá
				cận dưới $g$ của hàm mục tiêu của các phương án
				phát triển tiếp từ điểm 13.
				\item Nếu $g \geq f^*$ thì không phát triển tiếp
				lời giải từ điểm 13.
			\end{itemize}
		\end{column}
		
		% Cột bên phải: hình cây nhánh và cận
		\begin{column}{0.53\textwidth}
			\centering
			\includegraphics[width=1.15\linewidth]{branch_and_bound_tree.png}
		\end{column}
	\end{columns}
\end{frame}

\section{Bài toán người du lịch}
\begin{frame}[t]{BÀI TOÁN NGƯỜI DU LỊCH}
	\setlength{\leftmargini}{-1.2em}
	
	\begin{itemize}
		\item Phát biểu bài toán:
		\begin{itemize}
			\item Một người du lịch muốn đi tham quan $n$ thành phố $1, 2, \ldots, n$.
			\item \textit{Hành trình} là cách đi xuất phát từ thành phố 1 đi qua
			tất cả các thành phố còn lại, mỗi thành phố đúng một lần,
			rồi quay trở lại thành phố xuất phát 1.
			\item Biết $c(i,j)$ là chi phí đi từ thành phố $i$ đến thành phố $j$
			($i,j = 1,2,\ldots,n$).
			\item Tìm hành trình với tổng chi phí là nhỏ nhất.
		\end{itemize}
		
		\item Một số nhận xét:
		\begin{itemize}
			\item Số lượng hành trình của người du lịch là $(n-1)!$.
			\item Ta có tương ứng 1-1 giữa một hành trình của người du lịch:
			\[
			1 \to x[2] \to x[3] \to \ldots \to x[n] \to 1
			\]
			với một hoán vị $x = (x[2], x[3], \ldots, x[n])$ của $n-1$
			số tự nhiên $2,3,\ldots,n$.
			\item Chi phí hành trình:
			$\displaystyle f(x) = c(1,x[2]) + c(x[2],x[3]) + \ldots
			+ c(x[n-1],x[n]) + c(x[n],1).$
			
		\end{itemize}
	\end{itemize}
	
	% Ảnh + caption ghim bên phải, không ảnh hưởng khối text
	\placecontent{0.77\paperwidth}{0.32\paperheight}{0.24\paperwidth}{
		\centering
		\includegraphics[width=0.5\linewidth]{hamilton.png}\\[0.3em]
		 {\tiny\bfseries          % cỡ nhỏ + in đậm
		 	Sir William Rowan Hamilton [1]\\[-0.9em] % khoảng cách 2 dòng
		 	(1805--1865)
		 }
	}
	
\end{frame}


\begin{frame}[t,fragile]{BÀI TOÁN NGƯỜI DU LỊCH}
	\setlength{\leftmargini}{-1.2em}
	
	\begin{itemize}
		\item Giải bằng phương pháp duyệt toàn bộ:
		\begin{itemize}
			\item Hành trình:  $x = (1, x[2], x[3], \ldots, x[n], 1)$
		\end{itemize}
	\end{itemize}
	
	\vspace{0.3cm}  
	
	%===== HAI KHUNG CODE + CỘT CHỮ + MŨI TÊN =====
	\begin{tikzpicture}
		% Khung code bên trái (pseudo-code), sát mép trái
		\node (L) [inner sep=0pt, anchor=north west] at (0,0)
		{%
			\begin{minipage}[t]{0.40\textwidth}
				\begin{minted}[
					fontsize=\tiny,
					frame=single,
					framesep=1mm,
					xleftmargin=-20pt,
					numbers=none,
					autogobble=true,
					escapeinside=||
					]{text}
					|{\color{codeblue}try(k) \{ }||{\color{codegold}// thử các giá trị có thể gán cho x[k]}|
						|{\color{codeblue}for v in candidates(k) do \{}|
							|{\color{codeblue}if (check(v,k)) then \{}|
								|{\color{codeblue}x[k] = v;}|
									|{\color{red}[Update the data structure D]}|
									|{\color{codeblue}if (k == n) then solution();}|
									|{\color{codeblue}else try(k+1);}| 
									|{\color{red}[Recover the data structure D]}|
								|{\color{codeblue}\}}|
							|{\color{codeblue}\}}|
						|{\color{codeblue}\}}|
				\end{minted}
			\end{minipage}
		};
		
		% Khung code bên phải (C/C++), sát mép phải
		\node (R) [inner sep=0pt, anchor=north east] at (\textwidth,0)
		{%
			\begin{minipage}[t]{0.38\textwidth}
			\begin{minted}[
				fontsize=\tiny,
				frame=single,
				framesep=1mm,
				xrightmargin=-22pt,
				numbers=none,
				autogobble=true,
				escapeinside=|| % để chèn \color{red}
				]{text}
				Init:
				* f* = +∞; f = 0;  x[1] = 1;
				* for (int v = 2; v <= n; v++) visited[v] = 0;
				
				void Try(int k) {
					for (int v = 2; v <= n; v++) {
						if (!visited[v]) {
							x[k] = v;
							|{\color{red}visited[v] = 1;}|
							|{\color{red}f = f + c(x[k-1], x[k]);}|
							if (k == n) { // Update record
								int ftemp = f + c(x[n], x[1]);
								if (ftemp < f*) f* = ftemp;
							}
							else Try(k + 1);
							|{\color{red}f = f - c(x[k-1], x[k]);}|
							|{\color{red}visited[v] = 0;}|
						}
					}
				}
			\end{minted}
			
			\end{minipage}
		};
		
		% Cột chữ “Cần xác định …” đặt giữa hai khung, khoảng giữa chiều cao
		\coordinate (Astart) at ([yshift=1.5cm]L.south east);
		\coordinate (Aend)   at ([yshift=3.613cm]R.south west);
		\draw[->] (Astart) -- (Aend);
		\node[anchor=south west]   % góc dưới-trái của block chữ nằm tại toạ độ chỉ định
		at ($(Astart)+(0.3,-0.1cm)$) {   % dịch block chữ lên trên mũi tên 0.15cm
			\scriptsize
			\begin{tabular}{@{}l@{}}
				Cần xác định: \\
				1) \texttt{candidates(k)} \\
				2) \texttt{check(v,k)}
			\end{tabular}
		};

		
	\end{tikzpicture}
	
\end{frame}

\begin{frame}[t]{BÀI TOÁN NGƯỜI DU LỊCH}
	\setlength{\leftmargini}{-1.2em}
	
	\begin{itemize}
		\item[] \textbf{Giải bằng phương pháp nhánh và cận}
		\item[] {\color{HUSTRed}\textbf{Tính cận:}}
	\end{itemize}
	
	\begin{itemize}
		\item Ký hiệu 
		$c_{\min} = \displaystyle\min \{\, c(i,j) \mid i,j = 1,2,\ldots,n,\ i \neq j \,\}$
		là chi phí đi lại nhỏ nhất giữa các thành phố.
		
		\item Cận ước lượng chi phí hành trình đầy đủ cho nhánh hiện tại
		tương ứng với hành trình bộ phận $(1,u_2,\ldots,u_k)$ đã đi qua $k$
		thành phố:
		\[
		1 \rightarrow u_2 \rightarrow \cdots \rightarrow u_{k-1} \rightarrow u_k.
		\]
		
		\item Nếu cận dưới $g(1,u_2,\ldots,u_k) \ge f^*$ thì không đi tiếp từ
		hành trình bộ phận $(1,u_2,\ldots,u_k)$.
	\end{itemize}
\end{frame}

\begin{frame}[t]{BÀI TOÁN NGƯỜI DU LỊCH}
	\small                        % <-- thu nhỏ cỡ chữ một chút
	\setlength{\leftmargini}{-1.2em}
	
	\begin{itemize}
		\item Cận ước lượng chi phí hành trình đầy đủ cho nhánh hiện tại
		tương ứng với hành trình bộ phận $(1,u_2,\ldots,u_k)$ đã đi qua
		$k$ thành phố: $	1 \rightarrow u_2 \rightarrow \cdots \rightarrow u_{k-1} \rightarrow u_k$
		
		\vspace{-0.4em}            % bớt khoảng trống
		
		\begin{itemize}
			\item Chi phí phải trả theo hành trình bộ phận $(1,u_2,\ldots,u_k)$ là
			\[
			\sigma = c(1,u_2) + c(u_2,u_3) + \cdots + c(u_{k-1},u_k).
			\]
			
			\vspace{-0.4em}
			
			\item Để phát triển thành hành trình đầy đủ:
			\[
			\textcolor{HUSTRed}{1 \rightarrow u_2 \rightarrow \cdots \rightarrow u_{k-1}
				\rightarrow u_k}
			\rightarrow u_{k+1} \rightarrow u_{k+2} \rightarrow \cdots \rightarrow u_n
			\rightarrow 1
			\]
		\end{itemize}
		
		\vspace{-0.4em}
		
		\item Ta còn phải đi $n-k+1$ đoạn đường nữa, mỗi đoạn đường có chi phí
		không ít hơn $c_{\min}$, nên đoạn đường chưa đi có chi phí ít ra là
		$(n-k+1)c_{\min}$.
		
		\vspace{-0.2em}
		
		\item Vậy nếu đã đi hành trình bộ phận $(1,u_2,\ldots,u_k)$ thì đoạn đường
		còn lại dù đi thế nào thì tổng chi phí cũng lớn hơn hoặc bằng
		\( g(1,u_2,\ldots,u_k) = \sigma + (n-k+1)c_{\min}. \)
	\end{itemize}
\end{frame}


\begin{frame}[t,fragile]{BÀI TOÁN NGƯỜI DU LỊCH}
	\setlength{\leftmargini}{-1.2em}
	
	\begin{itemize}
		\item Hàm \texttt{Try(k)} tìm lời giải tối ưu cho bài toán người du lịch
		có sử dụng kỹ thuật {\color{HUSTRed}\textbf{nhánh}%
			\tikz[remember picture,baseline] \node (nhanh) {};%
			\textbf{ và cận}}.
	\end{itemize}
	

	
	%===== HAI KHUNG CODE, CẠNH DƯỚI TRÙNG NHAU =====
\centering % <-- Rất quan trọng để căn giữa cả hình

\begin{tikzpicture}[remember picture]
	%=== Khung bên trái: Main() ===
	\node (L) [inner sep=0pt, anchor=south east] at (-0.75,0)
	{%
		\begin{minipage}[t]{0.30\textwidth}
			\begin{minted}[
				fontsize=\tiny,
				frame=single,
				framesep=1mm,
				xleftmargin=-20pt,
				numbers=none,
				autogobble=true
				]{text}
				Main() {
					//Init:
					f* = +∞;  f = 0;  x[1] = 1;
					for v = 2 to n do  visited[v] = false;
					Try(2);
					print(f*);
				}
			\end{minted}
		\end{minipage}
	};
	
	%=== Khung bên phải: Try(k) ===
	\node (R) [inner sep=0pt, anchor=south west] at (0.75,0)
	{%
		\begin{minipage}[t]{0.30\textwidth}
			\begin{minted}[
				fontsize=\tiny,
				frame=single,
				framesep=1mm,
				xrightmargin=-22pt,
				numbers=none,
				autogobble=true,
				escapeinside=|| % để chèn \color{red}
				]{text}
				Try(k) {
					for v = 2 to n do {
						if not visited[v] {
							x[k] = v;
							visited[v] = true;
							f = f + c(x[k-1], x[k]);
							if k = n then { //Update record
								int ftemp = f + c(x[n], x[1]);
								if (ftemp < f*) f* = ftemp;
							}
					|{\fcolorbox{red}{white}{\ttfamily\color{red}\shortstack[l]{\hspace*{1.5em}else \{\\\hspace*{3em}g = f + (n-k+1)*cmin;\\\hspace*{3em}if g < f* then Try(k+1);\\\hspace*{1em}\}}}}|
							  
							
							f = f - c(x[k-1], x[k]);
							visited[v] = false;
						}
					}
				}
				
			\end{minted}
		\end{minipage}
	};
	
	\coordinate (elseTopRight) at
	($(R.north east)!0.46!(R.south east)$);
	
\end{tikzpicture}

\begin{tikzpicture}[remember picture,overlay]
	\draw[->,thick,HUSTRed]
	([xshift=-1.5em, yshift=0ex]nhanh.south) -- ([xshift=-0.8em]elseTopRight);
\end{tikzpicture}



	
\end{frame}





%Hết

{\HUSTUseBackground{theme_hust_oneside.pdf}
\begin{frame}
  \ifdefstring{\insertaspectratio}{169}{
    \placecontent{0.355\paperwidth}{0.410\paperheight}{0.640\paperwidth}{
        \color{HUSTRed}\bfseries\fontsize{28pt}{36pt}\selectfont\centering
        THANK YOU!
    }
  }{}
  \ifdefstring{\insertaspectratio}{43}{
    \placecontent{0.355\paperwidth}{0.440\paperheight}{0.640\paperwidth}{
        \color{HUSTRed}\bfseries\fontsize{28pt}{36pt}\selectfont\centering
        THANK YOU!
    }
  }{}
\end{frame}
}

\end{document}